\input{mmd6-article-leader}
\def\myauthor{Craig Hutchinson}
\def\mytitle{Flow-down of LASP PCU requirements to PCU-ID}
\def\mydate{02 April 2018}
\input{mmd6-article-begin}

\textbf{[Excel Row 1]}

\begin{quote}
The PCU shall include no limited life items that will last less than 1 year.
\end{quote}

[PCU.18005] Functional Lifetime of Parts

The PCU shall contain only items whose expected function life exceeds 14 months.

\begin{itemize}
\item{} Rationale: \emph{The SMRD RS.21040 specifies prime mission opereation period of 1 year following 2 months of commissioning activities.}

\end{itemize}

\textbf{[Excel Row 2]}

\begin{quote}
The PCU shall survive the allowable flight temperatures -30°C to +50°C (AFT is 10°C within proto-flight test range), measured at the mounting surface
\end{quote}

[PCU.18100] Allowable Flight Temperature

The PCU shall operate nominally between the temperatures -30 deg C and +50 deg C, measured at the mounting surface.

\begin{itemize}
\item{} Rationale: \emph{This represented the expected temperature regime on orbit in which the PCU needs to function.}

\end{itemize}

\textbf{[Excel Row 3]}

\begin{quote}
The PCU shall survive non-operational survival temperatures: -50°C to +75°C ( test margin not typically applied to survival range), measured at the mounting surface
\end{quote}

[PCU.18105] Non-Operational Survival Temperature

The PCU shall survive temperatures of -50 deg C to +75 deg C, measured at the mounting surface.

\begin{itemize}
\item{} Rationale: \emph{This represented the expected temperature regime on orbit to which the PCU may be exposed to and subsequently operate when returned to the allowable flight temperature.}

\end{itemize}

\textbf{[Excel Row 4]}

\begin{quote}
The PCU shall provide 28V nominal, with a range of 22V to 40V
\end{quote}

[PCU.12000] Output 28 VDC

The PCU shall output 28 VDC Power nominal on three power buses in accordance with LASP Document No. 154920.

\begin{itemize}
\item{} Rationale: \emph{The PCU generates 28 VDC and then splits it into three different nets (busses), which are routed to various sub-systems within the CPRSP.}

\end{itemize}

\textbf{[Excel Row 5]}

\begin{quote}
The PCU shall meet all functional and performance requirements after exposure to input voltages from 0 to 120V during brownout.
\end{quote}

[PCU.10005] Exposure to Brownout

The PCU shall operate nominally after exposure to input voltages from 0 to 120 VDC during brownout.

\begin{itemize}
\item{} Rationale: \emph{The ELC may provide less voltage than the advertised nominal value.}

\item{} Comments from CORE database: \textbf{Given the nominal input voltage range of 102V-126.5V, the brownout range cannot be all the way up to 120V. What is the source of this requirement? It seems like there should be more specifics to account for a particular condition. How does LASP want LaRC to verify this requirement?}

\end{itemize}

\textbf{[Excel Row 6]}

\begin{quote}
The PCU shall survive after exposure to an abrupt, unannounced removal of power
\end{quote}

[PCU.10010] Survival Following Unannounced Removal of Power

The PCU shall survive after exposure to an unplanned removal of power.

\begin{itemize}
\item{} Rationale: \emph{The ISS may remove 120V Operational Power from the ELC prior to notifying the CPOC due to operational constraints.}

\end{itemize}

\textbf{[Excel Row 7]}

\begin{quote}
The PCU EEE parts purchasing (selection and testing) shall be approved by the Part Control Borad as specified in the Parts Control Plan.
\end{quote}

\textbf{Requirement not flowed down: MAR, which specifies EEE parts purchasing, is applicable document for PCU ICD.}

\textbf{[Excel Row 8]}

\begin{quote}
The PCU shall be capable of meeting all functional and performance requirements at the total ionizing dose and trapped electron and proton flux levels in the ISS orbit.
\end{quote}

[PCU.18110] Total Ionizing Dose

The PCU shall operate nominally while exposed to the total ionizing dose levels as specified by SSP 30512.

\begin{itemize}
\item{} Rationale: \emph{The design of the PCU needs to account for the radiation environment that could threaten its operations.}\\
[PCU.18115] Trapped Electron Exposure

\end{itemize}

The PCU shall operate nominally while exposed to the trappeed electron flux levels as specified by SSP 30512.

\begin{itemize}
\item{} Rationale: \emph{The design of the PCU needs to account for the radiation environment that could threaten its operations.}\\
[PCU.18120] Trapped Proton Exposure

\end{itemize}

The PCU shall operate nominally while exposed to the trappeed proton flux levels as specified by SSP 30512.

\begin{itemize}
\item{} Rationale: \emph{The design of the PCU needs to account for the radiation environment that could threaten its operations.}

\end{itemize}

\textbf{[Excel Row 9]}

\begin{quote}
The PCU shall be designed to withstand Single Event Upsets without permanant damage.
\end{quote}

[PCU.18125] Single Event Upset

The PCU shall operate nominally without sustaining permanent damage following exposure to single event upsets as specified by SSP 30512.

\begin{itemize}
\item{} Rationale: \emph{The design of the PCU needs to account for the radiation environment that could threaten its operations.}

\end{itemize}

\textbf{[Excel Row 10]}

\begin{quote}
The PCU shall survive Single Event Latchup events without damage.
\end{quote}

[PCU.18130] Single Event Latchup

The PCU shall operate nominally without sustaining damage following exposure to single event latchup events as specified by SSP 30512.

\begin{itemize}
\item{} Rationale: \emph{The design of the PCU needs to account for the radiation environment that could threaten its operations.}

\end{itemize}

\textbf{[Excel Row 11]}

\begin{quote}
The PCU shall have 3 power outputs of 28 V nominal
\end{quote}

[PCU.12000] Output 28 VDC

The PCU shall output 28 VDC Power nominal on three power buses in accordance with LASP Document No. 154920.

\begin{itemize}
\item{} Rationale: \emph{The PCU generates 28 VDC and then splits it into three different nets (busses), which are routed to various sub-systems within the CPRSP.}

\end{itemize}

\textbf{[Excel Row 12]}

\begin{quote}
The PCU shall have built-in protection to prevent damage due to polarity reversal at the power inputs\slash outputs
\end{quote}

[PCU.10015] Protection from Input Polarity Reversal

The PCU shall protect itself from damage due to reversed polarity at the PCU 120 VDC input.

\begin{itemize}
\item{} Rationale: \emph{This prevents damage to the PCU should the connectors be attached improperly.}

\end{itemize}

\textbf{[Excel Row 13]}: GOLD Rule

\begin{quote}
The PCU shall comply with the structural analysis and design factors of safety defined in GEVS Section 2.2.5, table 2.2--2.
\end{quote}

[PCU.18200] Structural Analysis\slash Design Factors of Safety

The PCU shall comply with the structural analysis and design factors of safety, as defined in GSFC-STD-7000A.

\begin{itemize}
\item{} Rationale: \emph{GSFC-STD-7000A contains environmental verification standards for space flight hardware.}

\end{itemize}

\textbf{[Excel Row 14]}: Gold Rule

\begin{quote}
All threaded fasteners will employ a locking feature.
\end{quote}

[PCU.16110] Threaded Fasteners with Locking Features

The PCU shall use threaded fasteners that implement locking features.

\begin{itemize}
\item{} Rationale: \emph{The locking features decreases the risk that threaded fasteners will unseat themselves.}

\item{} Comments from CORE database: \textbf{Clarify intent with LASP}

\end{itemize}

\textbf{[Excel Row 15]}: Extravehicular Activity On-Orbit Induced Loads

\begin{quote}
External components the PCU exposed to Extravehicular Activity (EVA) crew shall maintain positive margins of safety for the loads defined in Table 3.1.3--1, Extravehicular Activity Induced Limit Loads.
SSP 57003 Table 3.1.3--1 Extravehicular Activity Induced Limit Loads (2 Pages)
\end{quote}

[PCU.18210] EVA Induced Loads

The PCU shall maintain positive margins of safety for any components exposed to EVA crew loads as defined in Table 3.1.3--1 of SSP 57003 Rev L.

\begin{itemize}
\item{} Rationale: \emph{Derives from Extravehicular Activity On-Orbit Induced Loads requirements of section 3.1.3 of SSP 57003 Rev L.}

\end{itemize}

\textbf{[Excel Row 16]}: Structural Materials Criteria and Selection

\begin{quote}
Mechanical properties and selection of PCU structural materials shall be in accordance with SSP 52005, Section 5.4.1, Allowable Mechanical Properties of Structural Materials.
Note: Stress Corrosion is addressed in SSP 51700, Section 3.9.3, Stress Corrosion.
\end{quote}

[PCU.18205] Structural Materials Criteria and Selection

The PCU shall incorporate structural materials in accordance with Section 5.4.1 of SSP 52005.

\begin{itemize}
\item{} Rationale: \emph{SSP 52005 is the Payload Flight Equipment Requirements and Guidelines for Safety-Critical Structures.}

\end{itemize}

\textbf{[Excel Row 17]}: Holes

\begin{quote}
Holes that are round or slotted, other than tether points, in the range of 0.5 to 1.4 inch (12.7 to 35.56 mm) shall be covered.
\end{quote}

[PCU.16115] Covered Holes

The PCU shall incorporate covers on all round or slotted holes in the range of 0.5 inch to 1.4 inch.

\begin{itemize}
\item{} Rationale: \emph{Covers prevent EVA crew from getting body parts stuck in holes. This derives from an EVA Human Engineering Safety requirement in section 3.8.3.2 of SSP 57003 Rev L.}

\end{itemize}

\textbf{[Excel Row 18]}: Burrs

\begin{quote}
Exposed surfaces shall be free of burrs.
\end{quote}

[PCU.16120] Burrs

The PCU shall have exposed surfaces free from burrs.

\begin{itemize}
\item{} Rationale: \emph{Burrs are a risk to the integrity of EVA crew spacesuits. This derives from an EVA Human Engineering Safety requirement in section 3.8.3.4 of SSP 57003 Rev L.}

\end{itemize}

\textbf{[Excel Row 19]}: Sharp Edges and Protrusion Criteria

\begin{quote}
Sharp edges and protrusions shall meet the criteria provided in Tables 3.8.3.7--1, Edge, Corner, and Protrusion Criteria - Edge and In-Plane Corner Radii, and Table 3.8.3.7--2, Edge, Corner, and Protrusion Criteria - Protrusions and Outside Corners, by the use of corner and edge guards or by rounding the corners and edges in accordance with Figure 3.8.3.7--1, Exposed Corner and Edge Requirements.
SSP 57003 Table 3.8.3.7--1 Edge, Corner, and Protrusion Criteria - Edge and In-Plane Corner Radii
SSP 57003 Table 3.8.3.7--2 Edge, Corner, and Protrusion Criteria - Protrusions and Outside Corners
SSP 57003 Figure 3.8.3.7--1 Exposed Corner and Edge Requirements
\end{quote}

[PCU.16125] Sharp Edges and Protrusion Criteria

The PCU shall meet sharp edge and protrusion criteria specified by Figure 3.8.3.7--1 and Tables 3.8.3.7--1 and 3.8.3.7--2 from SSP 57003.

\begin{itemize}
\item{} Rationale: \emph{Sharp edges and protrusions are a risk to the integrity of EVA crew spacesuits.}

\end{itemize}

\textbf{[Excel Row 20]}: On-Orbit Loads

\begin{quote}
PCU shall be designed to maintain positive structural margins of safety when exposed to the loading induced by plate deformations in bending
\end{quote}

[PCU.18215] On-Orbit Loads

The PCU shall maintain positive structural margins of safety when exposed to the loading induced by plate deformations in bending.

\begin{itemize}
\item{} Rationale: \emph{LASP original: ``PCU shall be designed to maintain positive structural margins of safety when exposed to the loading induced by plate deformations in bending''}

\item{} Comments from CORE database: \textbf{\textbf{Unsure what this means. Follow up with LASP}}

\end{itemize}

\textbf{[Excel Row 21]}: Gold Rule

\begin{quote}
All electrical, electronic, and electro-mechanical components will be subjected to minimum workmanship test levels as specified in GEVS Section 2.4.2.5.
\end{quote}

[PCU.18220] EEE Minimum Workmanship Vibroacoustic Test Levels

The PCU shall meet minimum workmanship test levels specified by Section 2.4.2.5 of GSFC-STD-7000A.

\begin{itemize}
\item{} Rationale: \emph{Section 2.4.2.5 of GSFC-STD-7000A is Compoment\slash Unit Vibroacoustic Tests section of GEVS}

\end{itemize}

\textbf{[Excel Row 22]}: GOLD Rule

\begin{quote}
All avionics enclosures will be protected from ESD. All external connectors must be fitted with shorting plugs or appropriate caps during transportation between locations. Additionally, all test points and plugs must be capped or protected from discharge for flight.
\end{quote}

\textbf{Requirement not flowed down: Look like delivery conditions as opposed to functional requirements. Research GOLD rules and determine whether this is a process or a box requirement. Text not in GEVS or SSP 57003}

\textbf{[Excel Row 23]}: EPS Circuit Protection - Interface C

\begin{quote}
The Payload connected to Interface C (defined in SSP 57003 Figure 3.2.2--2, Interface C, including payload sites and robotic interfaces) circuit protection device shall be designed to provide trip coordination, i.e., the downstream circuit protection device disconnects a shorted circuit or an overloaded circuit from the upstream power interface, without tripping the upstream circuit protection device. The trip coordination is achieved either by a shorter trip time or lower current limitation than the upstream protection devices described in Table 3.2.2--1, Detailed Upstream Protection Characteristics, Tables 3.2.2--2, PVGF to User Electrical Interface Parameters, and Figure 3.2.2--1, ITS S3\slash P3 and MCAS Overload Protection Characteristics.
Note: Current limiting protection devices start to limit the current when the current reaches the limiting threshold. The shaded regions in the figures show the current limit regions from the time the protection devices start to control the current within the specified range to the maximum time where the protection device trips and interrupts the current flow. Nominal current ratings are 25 amperes. The current at the S3\slash P3 APPI will be controlled to within the limiting level of 27.5 to 30 amperes within 1 millisecond. The current at the MCAS power interface will be controlled to within the limiting level of 13.2 to 14.4 amperes within 1 millisecond. The Remote Power Controller (RPC) will trip if the current remains in the limiting region up to the decision time of 34.5 +\slash - 3.5 milliseconds.
\end{quote}

[PCU.14020] Circuit Protection Trip Coordination

The PCU shall provide trip coordination such that sub-system over-current protection nearest to the fault will activate prior to over-current protection further upstream.

\begin{itemize}
\item{} Rationale: \emph{This derives from the Electrical Power System Circuit Protection Characteristics requirements in section 3.2.2.A of SSP 57003 Rev L.}

\end{itemize}

\textbf{[Excel Row 24]}: EPS Circuit Protection - Overcurrent

\begin{quote}
Overcurrent protection shall be provided at all points in the system where power is distributed to lower level (wire size not protected by upstream circuit protection device) feeder and branch lines in compliance with SSP 57003 Table 3.2.2--5, Ambient Operating Temperature for Derating Relays and Switches, Table 3.2.2--6, Cycle Rate per Hour for Derating Relays and Switches, and Table 3.2.2--7, Load Application Rate for Derating Relays and Switches.
\end{quote}

[PCU.14025] Circuit Protection for Overcurrent

The PCU shall provide overcurrent protection at all points where power is distributed to lower level feeder and branch lines as specified by Tables 3.2.2--3, 3.2.2--4, 3.2.2--5, 3.2.2--6, and 3.2.2--7 of SSP 57003 Rev L.

\begin{itemize}
\item{} Rationale: \emph{nan}

\end{itemize}

\textbf{[Excel Row 25]}: Payload Power Isolation

\begin{quote}
The PCU power shall be DC isolated from chassis, structure, equipment conditioned power return\slash reference, and signal returns by a minimum of 1 MegaOhm.
\end{quote}

[PCU.15100] Payload Power Isolation

The PCU shall isolate its DC power from chassis, structure, equipment conditioned power return\slash reference, and signal returns by a minimum of 1 MegaOhm.

\begin{itemize}
\item{} Rationale: \emph{This derives from the Payload Power Isolation requirements in section 3.2.3.3.1 of SSP 57003 Rev L.}

\end{itemize}

\textbf{[Excel Row 26]}: Single Point Ground

\begin{quote}
Each isolated electrical power source shall be connected to structure at no more than one point.
\end{quote}

[PCU.15105] Single Point Ground

The PCU shall connect each electrically isolated power source to the structure at no more than one point.

\begin{itemize}
\item{} Rationale: \emph{This derives from the Single Point Ground requirement in section 3.2.3.3.2.1 of SSP 57003 Rev L.}

\end{itemize}

\textbf{[Excel Row 27]}: Electrical Power Isolation

\begin{quote}
Each isolated electrical power source shall be DC isolated from chassis, structure, equipment conditioned power return\slash reference, and signal circuits by a minimum of 1 MegaOhm, individually, except at the single point ground.
\end{quote}

[PCU.15110] Electrical Power Isolation

The PCU shall isolate each isolated electrical DC power source from chassis, structure, equipment-conditioned power return\slash reference, and signal circuits by a minimum of 1 MegaOhm, individually, except at the single point ground.

\begin{itemize}
\item{} Rationale: \emph{This derives from the Electrical Power Isolation requirements in section 3.2.3.3.2.3 of SSP 57003 Rev L.}

\end{itemize}

\textbf{[Excel Row 28]}: Signal Circuit Returning Grounding

\begin{quote}
Circuit conductors shall be DC isolated from chassis, structure, and equipment conditioned power return\slash reference, by a minimum of 1 MegaOhm, individually, when not terminated by the signal circuit's single point ground\slash reference. Balanced, differential circuits isolated from chassis, structure, and user conditioned power return\slash reference by a minimum of 6000 ohms complies with this requirement.
Note: Signals circuits with frequency components equal to or above 4 megahertz may use controlled impedance transmission and reception media such as (but not limited to):
\end{quote}

\begin{itemize}
\item{} shielded twisted 72 ohm cable

\item{} ``twin ax'' cable balanced and referenced to primary structure at a single point

\item{} ``triax'' cable using the center and inner shield conductors for unbalanced transmission, referenced to primary structure at a single point with the outer shield multipoint grounded as an ``overshield''

\item{} ``coax'' cable with the shield terminated 360 degrees at each end at available intermediate point (permitted for signals with the lowest frequency component equal to or above 4 MHz).

\end{itemize}

[PCU.15115] Signal Circuit Return Grounding

The PCU shall DC isolate circuit conductors from chassis, structure, and equipment conditioned power return\slash reference, by a minimum of 1 MegaOhm, individually, when not terminated by the signal circuit's single point ground\slash reference.

\begin{itemize}
\item{} Rationale: \emph{Balanced, differential circuits isolated from chassis, structure, and user conditioned power return\slash reference by a minimum of 6000 ohms complies with this requirement. Note: Signals circuits with frequency components equal to or above four megahertz may use controlled impedance transmission and reception media such as (but not limited to): - shielded twisted 72 ohm cable - ``twin ax'' cable balanced and referenced to primary structure at a single point - ``triax'' cable using the center and inner shield conductors for unbalanced transmission, referenced to primary structure at a single point with the outer shield multipoint grounded as an ``overshield''. - ``coax'' cable with the shield terminated 360 degrees at each end and at available intermediate point (permitted for signals with the lowest frequency component equal to or above 4 MHz). This derives from the Signal Circuit Returning Grounding requirement in section 3.2.3.3.2.3 of SSP 57003 Rev L.}

\end{itemize}

\textbf{[Excel Row 29]}: Analog, Differential Circuit Return

\begin{quote}
Each differential analog circuit shall employ a separate return.
\end{quote}

[PCU.15120] Analog, Differential Circuit Return

The PCU shall provide a separate return for each differential analog circuit.

\begin{itemize}
\item{} Rationale: \emph{This derives from the Analog, Differential Circuit Return requirement in section 3.2.3.3.2.5 of SSP 57003 Rev L.}

\end{itemize}

\textbf{[Excel Row 30]}: Discrete Returns

\begin{quote}
Low level discrete signals shall use individual returns.
\end{quote}

[PCU.11015] Discrete Returns

The PCU shall provide an individual return for each ISS low level discrete signal.

\begin{itemize}
\item{} Rationale: \emph{This derives from the Analog, Differential Circuit Return requirement in section 3.2.3.3.2.6 of SSP 57003 Rev L.}

\end{itemize}

\textbf{[Excel Row 31]}: CS06 Limits

\begin{quote}
Payloads with safety-critical circuits connecting to a 120 VDC ISS source shall not (and other payload circuits should not) exhibit any malfunction, degradation of performance, or deviation from specified indications beyond the tolerances indicated in the individual equipment or subsystem specification when the test spikes, each having the waveform shown on Figure 3.2.4.2.3--1, CS06 Equipment Limit, are applied to the DC power input leads.
Note: CS06 is applicable to DC power leads of equipment are subsystems powered by 120 VDC sources, including grounds and returns, which are not grounded internally to the equipment or subsystem. (From SSP 30237, 3.2.2.3)
Note: Both the positive and negative spike will be applied to each lead under test. each spike will be applied one time per lead instead of multiple spikes over the course of two minutes. The values of E and t are given below.
SSP 57003 Figure 3.2.4.2.3--1 CS06 Equipment Limit
\end{quote}

[PCU.19005] Conducted Susceptibility (CS06) Limits

The PCU shall prevent any malfunction, degradation of performance, or deviation that exceeds tolerances when subjected to test spikes , in accordance with Figure 3.2.4.2.3--1 of SSP 57003 Rev L, applied to DC power input leads.

\begin{itemize}
\item{} Rationale: \emph{Note: CS06 is applicable to dc power leads of equipment and subsystem powered by 120 Vdc sources, including grounds and returns, which are not grounded internally to the equipment or subsystem. (From SSP 30237, 3.2.2.3) Note: Both the positive and negative spike will be applied to each lead under test. Each spike will be applied one time per lead instead of multiple spikes over the course of two minutes. This derives from the Conducted Susceptibility (CS06) Limits requirements in section 3.2.4.2.3 of SSP 57003 Rev L}

\end{itemize}

\textbf{[Excel Row 32]}: Shock Hazard - Class H, Static Bond - Class S (reworded for flow down)

\begin{quote}
The PCU shall comply with the coating requirement for mating surfaces, such that the mechanical bonding surface between the PCU and the ExPA is conductive (conversion coated) to MIL-DTL-5541 class 3 Type 1 (SSP 30245 Revision P).
\end{quote}

[PCU.19105] Shock Hazard - Class H, Static Bond - Class S

The PCU shall comply with the coating requirement for mating surfaces, such that the mechanical bonding surface between the PCU and the ExPA is conductive (conversion coated) to MIL-DTL-5541 class 3 Type 1 (SSP 30245 Revision P).

\begin{itemize}
\item{} Rationale: \emph{nan}

\end{itemize}

\textbf{[Excel Row 33]}: Current Draw

\begin{quote}
CPRSP shall limit the continuous current drawn from each EOTP power bus to a maximum of 1.6 A (minimum heater resistance of 78 ohm resulting from 124.6 V\slash 1.6 A).
Note: The maximum combined current for all SPDM payloads (attached to LEE, both OTCMs, and bath EOTP PFRAM interfaces) is limited to 8.2 A.
Note: The EOTP electrical interface is only defined for resistive heater power loads. Reactive or active power loads (ex. motors or avionics) are not supported unless they can be shown to meet the current draw requirement with appropriate inhibits (such as the inability to activate due to no command path).
\end{quote}

[PCU.14100] EOTP Current Draw

The PCU shall not draw current from the EOTP.

\begin{itemize}
\item{} Rationale: \emph{The EOTH electrical interface is only defined for resistive heater power loads. Non-resistive electrical logs, \emph{e.g.} motors, computers, capacitors, \emph{etc.}, are not supported by the EOTP. Payloads must have inhibits in place for all non-resistive electrical loads such that no power is being drawn from the EOTP power buses when they are energized (SSP 57003, 3.7.3.5.3). ExPA-based Payloads need to be aware that the EOTP Primary Heater Power (120 VDC Bus 1) use the same lines as the ELC operational power lines. Note that the EOTP has no command or telemetry for payloads.}

\end{itemize}

\textbf{[Excel Row 34]}: EOTP Electrical Loads

\begin{quote}
CPRSP shall present only resistive power loads to the EOTP electrical interface or shall show that appropriate inhibits (such as the inability to activate given no command path) are in place to prevent a reactive or active power load (ex. motors or avionics) from drawing power from EOTP.
\end{quote}

[PCU.14100] EOTP Current Draw

The PCU shall not draw current from the EOTP.

\begin{itemize}
\item{} Rationale: \emph{The EOTH electrical interface is only defined for resistive heater power loads. Non-resistive electrical logs, \emph{e.g.} motors, computers, capacitors, \emph{etc.}, are not supported by the EOTP. Payloads must have inhibits in place for all non-resistive electrical loads such that no power is being drawn from the EOTP power buses when they are energized (SSP 57003, 3.7.3.5.3). ExPA-based Payloads need to be aware that the EOTP Primary Heater Power (120 VDC Bus 1) use the same lines as the ELC operational power lines. Note that the EOTP has no command or telemetry for payloads.}

\end{itemize}

\textbf{[Excel Row 35]}: Fault Protection

\begin{quote}
If the CPRSP connects to the EOTP power circuits, the user shall ensure a design such that internal faults will cause the faulted portion to disconnect from the power circuit with maximum overload interrupt times less than the times defined in Table 3.7.3.5.4--1, Maximum Overload Interrupt Times for User Fault Protection, and Figure 3.7.3.5.4--1, Maximum Overload Interrupt Times for User Fault Protection, to ensure protection compatibility with EOTP fuse. Fault in this context is a condition under which the user is consuming excessive current, as defined by the developer, but not to exceed 150 percent of normal operating current.
Note: Payloads containing resistive heater systems controlled by bimetallic thermostats or other mechanical electric switches do not need to verify fault protection per interpretation of EPS-TIA-076.
SSP 57003 Table 3.7.3.5.4--1 Maximum Overload Interrupt Times for User Fault Protection
SSP 57003 Figure 3.7.3.5.4--1 Maximum Overload Interrupt Times for User Fault Protection
\end{quote}

[PCU.14100] EOTP Current Draw

The PCU shall not draw current from the EOTP.

\begin{itemize}
\item{} Rationale: \emph{The EOTH electrical interface is only defined for resistive heater power loads. Non-resistive electrical logs, \emph{e.g.} motors, computers, capacitors, \emph{etc.}, are not supported by the EOTP. Payloads must have inhibits in place for all non-resistive electrical loads such that no power is being drawn from the EOTP power buses when they are energized (SSP 57003, 3.7.3.5.3). ExPA-based Payloads need to be aware that the EOTP Primary Heater Power (120 VDC Bus 1) use the same lines as the ELC operational power lines. Note that the EOTP has no command or telemetry for payloads.}

\end{itemize}

\textbf{[Excel Row 36]}: Component Hazardous Energy Provision

\begin{quote}
Components which retain hazardous energy potential shall either be designed to prevent a crewmember from releasing the stored energy potential or be designed with provisions to allow safing of the potential energy, including provisions to confirm that the safing was successful.
\end{quote}

[PCU.16130] Component Hazardous Energy Provision

The PCU shall prevent a crewmember from releasing any hazardous energy potential.

\begin{itemize}
\item{} Rationale: \emph{This derives from an EVA Human Engineering Safety requirement in section 3.8.3.9 of SSP 57003 Rev L.}

\end{itemize}

\textbf{[Excel Row 37]}: 120 VDC Normal Steady-State Voltage Requirements

\begin{quote}
The ELC provides steady-state voltage within the limits of 106.5 VDC to 126.5 VDC to ELC Payloads connected to ELC 120 VDC heater or operational power.
\end{quote}

[PCU.10000] Input 120 VDC

The PCU shall operate nominally given a steady-state voltage between 106.5 VDC and 126.5 VDC on the Operational Power input in accordance with LASP Document No. 154920.

\begin{itemize}
\item{} Rationale: \emph{The PCU converts the 120 VDC Operational Power supplied by the ExPA into 28 VDC. Document No. 154920 identifies the pins, part number, manufacturer, and description of the 120V Power Input interface jack and mating connector. This derives from the ELC 120 VDC Power Interface requirements of section E.3.2.1.1.1 of SSP 57003 Rev L. Figure E.3.2.1.1.2--1 implies that the PCU could see a minimum of 102 VDC.}

\end{itemize}

\textbf{[Excel Row 38]}: 120 VDC Non-Normal Voltage Transients

\begin{quote}
CPRSP should not be damaged with the transient voltage conditions that are within the limits shown in Figure E.3.2.1.1.2--1, 120 Vdc Fault Clearing and Protection Transient Limits. Loads may be exposed to transient overvoltage conditions during operation of the power system’s fault protection components.
Figure E.3.2.1.1.2--1 120 VDC Fault Clearing and Protection Transient Limits
\end{quote}

[PCU.10020] 120 VDC Non-Normal Voltage Transients

The PCU shall operate nominally following exposure to transient voltage conditions as specified by SSP 57003 Rev L Figure E.3.2.1.1.2--1.

\begin{itemize}
\item{} Rationale: \emph{Loads may be exposed to transient overvoltage conditions during operation of the power system's fault protection components. This derives from the ELC 120 VDC Power Interface requirements of section E.3.2.1.1.2 of SSP 57003 Rev L.}

\end{itemize}

\textbf{[Excel Row 39]}: Reverse Current

\begin{quote}
CPRSP shall limit reverse current transients that can occur when a hard fault occurs across the power source within the transient envelope defined for 25 A curves shown in Figures E.3.2.1.1.4--1, Reverse Current Envelopes for Time Duration Shorter than 300 Microseconds, E.3.2.1.1.4--2, Reverse Current Envelopes for Time Duration Between 300 Microseconds and 10 Milliseconds, and E.3.2.1.1.4--3, Reverse Current Envelopes for Time Duration Between 10 and 300 Milliseconds.
For purposes of this interface definition, the fault is 10 milliohms or less applied within 2 micro−seconds or less. For the ELC Payload\slash payload exhibiting reverse current transient peaks within +\slash −100 Amperes, the fault resistance is 40 milliohms or less.
SSP 57003 Figure E.3.2.1.1.4--1 Reverse Current Envelopes for Time Duration Shorter than 300 microseconds
SSP 57003 Figure E.3.2.1.1.4--2 Reverse Current Envelopes for Time Duration Between 300 Microseconds and 10 Milliseconds
SSP 57003 Figure E.3.2.1.1.4--3 Reverse Current Envelopes for Time Duration Between 10 and 300 Milliseconds
\end{quote}

[PCU.10025] Reverse Current

The PCU shall limit reverse current transients in accordance with Figures E.3.2.1.1.4--1, E.3.2.1.1.4--2, E.3.2.1.1.4--3 of SSP 57003 Rev L when exposed to a hard fault across the power source, as specified by section E.3.2.1.1.4 of SSP 57003 Rev L.

\begin{itemize}
\item{} Rationale: \emph{For purposes of this interface definition, the fault is 10 milliohms or less applied within 2 micro-seconds or less. For the ELC Payload\slash payload exhibiting reverse current transient peaks within +\slash -100 Amperes, the fault resistance is 40 milliohms or less. This derives from the ELC 120 VDC Power Interface requirements of section E.3.2.1.1.4 of SSP 57003 Rev L.}

\end{itemize}

\textbf{[Excel Row 40]}: 120 VDC Operational Power Ripple Voltage and Noise

\begin{quote}
The PCU, connected to ELC 120 Vdc output, shall operate and be compatible with the EPS time domain ripple voltage and noise level of 2.5 Vrms maximum within the frequency range of 30 Hz to 1 MHz.
\end{quote}

[PCU.10030] 120 VDC Operational Power Ripple Voltage and Noise

The PCU shall operate and be compatible with the EPS time domain ripple voltage and noise level of 2.5 Vrms maximum within the frequency range of 30 Hz to 1 MH when connected to 120 VDC Operational Power.

\begin{itemize}
\item{} Rationale: \emph{Derives from the ELC 120 VDC Power Interface requirements of section E.3.2.1.1.6.2 of SSP 57003 Rev L.}

\end{itemize}

\textbf{[Excel Row 41]}: 120 VDC Operational Power Ripple Voltage Spectrum

\begin{quote}
The PCU, connected to ELC 120 Vdc output, shall operate and be compatible with the EPS ripple voltage spectrum of a maximum as shown in Figure E.3.2.1.1.6.3--1, 120 Vdc Maximum Ripple Voltage Spectrum.
Note: This limit is 6 decibels (dB) below the Electromagnetic Compatibility (EMC) CS-01, 02 requirements in SSP 30237, Space Station Requirements for Electromagnetic Emission and Susceptibility Requirements, up to 30 MHz.
SSP 57033 Figure E.3.2.1.1.6.3--1 120 VDC Maximum Ripple Voltage Spectrum
\end{quote}

[PCU.10035] 120 VDC Operational Power Ripple Voltage Spectrum

The PCU, connected to ELC 120 VDC output, shall operate and be compatible with the EPS ripple voltage spectrum of a maximum as shown in SSP 57003 Rev L Figure E.3.2.1.1.6.3--1.

\begin{itemize}
\item{} Rationale: \emph{Note: This limit is 6 decibels (dB) below the Electromagnetic Compatibility (EMC) CS-01, 02 requirement in SSP 30237, Space Station Requirements for Electromagnetic Emission and Susceptibility Requirements, up to 30 MHz. Derives from the ELC 120 VDC Power Interface requirements of section E.3.2.1.1.6.3 of SSP 57003 Rev L.}

\end{itemize}

\textbf{[Excel Row 42]}: 120 VDC Operational Power Surge Current

\begin{quote}
The surge current at the power input to the PCU shall not exceed the protective device trip setting characteristics as defined in Table E.3.2.1.1.6.4--1, ISS 120 Vdc Overload Protection Characteristics.
SSP 57003 Table E.3.2.1.1.6.4--1 ISS 120 VDC Overload Protection Characteristics
\end{quote}

[PCU.10040] 120 VDC Operational Power Surge Current

The PCU shall limit in-rush current at power input connected to ELC 120 VDC output interface in accordance with Table E.3.2.1.1.6.4--1 of SSP 57003 Rev L.

\begin{itemize}
\item{} Rationale: \emph{Derives from the ELC 120 VDC Power Interface requirements of section E.3.2.1.1.6.4 of SSP 57003 Rev L.}

\end{itemize}

\textbf{[Excel Row 43]}: Contamination Control, Planning, and Execution

\begin{quote}
Specific contamination control requirements and processes (such as analytical modeling, laboratory investigations, and contamination protection and avoidance plans) that support mission objectives will be identified.
\end{quote}

\textbf{Requirement not flowed down: Not a PCU requirement.}

\textbf{[Excel Row 44]}: Pressure and Humidity

\begin{quote}
CPRSP will be exposed to an on-orbit minimum pressure environment of 5.5x10\^{}-12 pounds per square inch absolute (psia) (2.8x10--10 Torr). This is to be used for design and analysis purposes. The very low pressure and zero percent relative humidity on-orbit can cause problems for materials and components that are not vacuum stable. Outgassing of components and materials can lead to performance degradation and eventual failure of the material or component. Improperly sealed assemblies or components that vent over time may eventually fail due to corona and arcing effects.
\end{quote}

\textbf{Requirement not flowed down: Not a PCU requirement, but a statement of expected environment.}

\textbf{[Excel Row 45]}: Atomic Oxygen (AO)

\begin{quote}
A. CPRSP will be exposed to a long term average ram atomic oxygen flux of 5.0x10\textsuperscript{21} atoms per cm\textsuperscript{2} per year for the on-orbit exposure duration. This is to be used for design and analysis purposes. B. Surfaces exposed 30 days or less will be exposed to a short term peak ram atomic oxygen flux of up to 4.4x10\textsuperscript{19} atoms per cm\textsuperscript{2} per day. This is to be used for design and analysis purposes. Note: Atomic oxygen exposure can cause oxidation of metals such as silver, erosion of plastics and other organic materials and changes in physical properties (i.e. converting silicone adhesives and thermal coatings to brittle silicates).
\end{quote}

\textbf{Requirement not flowed down: empty}

\textbf{[Excel Row 46]}: Cleanliness

\begin{quote}
CPRSP hardware external surfaces shall conform to the Visible Clean-Sensitive (Vc-S) cleanliness level (See Glossary of Terms) upon delivery.
\end{quote}

\textbf{Requirement not flowed down: Not germane to PCU}

\textbf{[Excel Row 47]}: Molecular Deposition from Materials Outgassing and Venting

\begin{quote}
A.CPRSP's component materials exposed to space vacuum (which includes any internal materials within a non−pressurized shell as well as external materials) and vents shall limit cumulative contaminant deposit on other Payloads, using the nominal operating temperature of the contamination source materials (emitters) and nominal operating temperatures of other Attached Payloads (receivers), to less than the following specifications:
\end{quote}

\begin{itemize}
\item{} Single Payload on an ELC: 5.0×10--15 gm\slash cm2・sec [\textsubscript{15} Å\slash year]

\item{} Truss-mounted payload: 1.0×10--14 gm\slash cm2・sec [\textsubscript{30} Å\slash year]

\item{} Airlock deployable payload: \textsubscript{1} Å\slash deployment.

\end{itemize}

\begin{quote}
B. CPRSP's component materials exposed to space vacuum (which includes any internal materials within a non−pressurized shell as well as external materials) and vents shall limit cumulative contaminant deposit on ISS elements, using the nominal operating temperature of the contamination source materials (emitters) and nominal operating temperatures of other Payloads (receivers), to less than the following specifications:
\end{quote}

\begin{itemize}
\item{} Single Payload on an ELC: 5.0×10--16 gm\slash cm2・sec [\textsubscript{1}.5 Å\slash year]

\item{} Truss-mounted payload: 1.0×10--15 gm\slash cm2・sec [\textsubscript{3} Å\slash year]

\item{} Airlock deployable payload: \textsubscript{1} Å\slash deployment.

\end{itemize}

\begin{quote}
C. Data Deliverable: Provide a preliminary characterization of CPRSP contaminant sources which will support a preliminary analysis of 3.5.3.2.A and 3.5.3.2.B by the Program. The preliminary analysis by the Program will identify potential induced contamination issues to the Payload. The PD is responsible for providing updates to the Program if sources of contamination are added or modified for assessment by the Program and for identification of issues.
Materials data delivery:
\end{quote}

\begin{itemize}
\item{} characterize payload materials outgassing sources of contamination (per Program supplied delivered template) and include identification of non-metallic materials exposed to vacuum (all materials outside of a pressurized or hermetically sealed volume, with a vacuum exposed surface area of 0.1 m2 or greater), [Typical
MIUL data deliveries do not contain all the required data are inadequate to characterize contamination sources.]

\item{} vacuum exposed surface areas,

\item{} predicted on-orbit operating temperature data (maximum, minimum and percentages of time spent between maximum to 60 °C, 60 °C to 30 °C, and less than 30 °C) [percentages of time spent at the specified temperature ranges are not required for airlock deployable payloads], a data template is available to guide
development of the data deliverable, and

\item{} predicted on-orbit operating temperature data (including maximum, minimum, and percentages of time spent at the Program specified temperature ranges, per template) [percentages of time spent at the Program specified temperature ranges are not required for airlock deployable payloads], and

\item{} outgassing rate data according to the ASTM E1559 standard, Method B, per ISS specification [if determined by the Program that a material is not covered by available data]. [Not required for airlock deployable payloads.] ASTM E595. Thermal Vacuum Stability data is not acceptable in place of ASTM E1559 outgassing rate data. Typical MIUL data deliveries do not contain all the required data are inadequate to characterize contamination sources.

\end{itemize}

\begin{quote}
Vacuum venting data delivery:
\end{quote}

\begin{itemize}
\item{} characterize payload vacuum venting sources of contamination [per Program supplied delivered template] and

\item{} include characterization of vent location, direction vectors, mass flow rate, effluent composition (including traces), pressure, temperature, duration and frequency of operations.

\end{itemize}

\begin{quote}
D. Data Deliverable: Provide the final characterization of payload contamination sources to the Program to support the final analysis of 3.5.3.2.A and 3.5.3.2.B by the Program. The final verification analysis will be conducted by the Program and the analysis results are to be used by the Payload\slash Program to complete the verification.
Materials data delivery:
\end{quote}

\begin{itemize}
\item{} characterize payload materials outgassing sources of contamination (per Program supplied delivered template) and include identification of non-metallic materials exposed to vacuum (all materials outside of a pressurized or hermetically sealed volume, with a vacuum exposed surface area of 0.1 m2 or greater),

\item{} vacuum exposed surface areas,

\item{} predicted on-orbit operating temperature data (maximum, minimum and percentages of time spent between maximum to 60 °C, 60 °C to 30 °C, and less than 30 °C) [percentages of time spent at the specified temperature ranges are not required for airlock deployable payloads], a data template is available to guide development of the data deliverable, and

\item{} predicted on-orbit operating temperature data (including percentages of time spent at the Program specified temperature ranges, per template), and

\item{} outgassing rate data according to the ASTM E1559 standard, Method B, per ISS specification [if determined by the Program that a material is not covered by available data]. ASTM E595 Thermal Vacuum Stability data is not acceptable in place of ASTM E1559 outgassing rate data. Typical MIUL data deliveries do not contain all the required data are inadequate to characterize contamination sources.

\end{itemize}

\begin{quote}
Vacuum venting data delivery:
\end{quote}

\begin{itemize}
\item{} characterize payload vacuum venting sources of contamination [per Program supplied delivered template] and

\item{} include characterization of vent location, direction vectors, mass flow rate, effluent composition (including traces), pressure, temperature, duration and frequency of operations.

\end{itemize}

[PCU.16205] Molecular Deposition onto Other Payloads

The PCU shall limit cumulative contaminant deposit on other Payloads, using the nominal operating temperature of the contamination source materials (emitters) and nominal operating temperatures of other Attached Payloads (receivers), to less than \TBD{Molecular Deposition onto Other Payloads Limit} g\slash cm\textsuperscript{2}\slash s.

\begin{itemize}
\item{} Rationale: \emph{Derives from External Contamination Requirements of section 3.5.3.2.A of SSP 57003 Rev L.}\\
[PCU.16210] Molecular Deposition onto ISS Elements

\end{itemize}

The PCU shall limit cumulative contaminant deposit on ISS elements, using the nominal operating temperature of the contamination source materials (emitters) and nominal operating temperatures of other Payloads (receivers), to less than \TBD{Molecular Deposition onto Other ISS Elements Limit} g\slash cm\textsuperscript{2}\slash s.

\begin{itemize}
\item{} Rationale: \emph{Derives from External Contamination Requirements of section 3.5.3.2.B of SSP 57003 Rev L.}

\end{itemize}

\textbf{[Excel Row 48]}: Particulates

\begin{quote}
CPRSP shall limit any active venting release of particulates to less than 100 microns in size.
\end{quote}

[PCU.16215] Particulates

The PCU shall limit any activate venting release of particulates to less than 100 microns in size.

\begin{itemize}
\item{} Rationale: \emph{Derives from External Contamination Requirements of section 3.5.3.3 of SSP 57003 Rev L.}

\end{itemize}

\textbf{[Excel Row 49]}: Material Breakdown Voltages

\begin{quote}
A. CPRSP's external hardware shall ensure that materials (including surface coatings) that are directly exposed to the defined plasma environment have breakdown voltages in excess of +20\slash -80 volts for hardware located inboard of the Solar Alpha Rotary Joint (SARJ) and +20\slash -90 volts for hardware located outboard and including the SARJ. This requirement affects the payload’s vacuum exposed surfaces that are in direct contact with
the plasma environment.
Note: The ISS is charged by the plasma environments interaction with the solar arrays and the magnetic induction of the geomagnetic field. Because of this charging, payloads can experience the specified potentials. It is necessary to determine whether the payload will arc at these potentials, as arcing would affect the payload thermal properties.
B. Data Deliverable: Provide a preliminary characterization of external surfaces (dielectric surfaces) by the Payload which will support a preliminary analysis of 3.5.4.1.A by the Program. The preliminary analysis by the Program will identify potential material breakdown voltage issues to the Payload. The PD is responsible for providing updates to the Program if the input data changes.
C. Data Deliverable: Provide a final characterization of payload external surfaces (dielectric surfaces) to the Program to support the final analysis by the Program. The final verification analysis will be conducted by the Program (ISS Space Environments Team) and the analysis results will be used by the Payload\slash Program to complete the verification.
\end{quote}

[PCU.16300] Material Breakdown Voltages

The PCU shall incorporate materials, including surface coatings, with breakdown voltages in excess of +20\slash -80 volts on all components directly exposed to the plasma environment.

\begin{itemize}
\item{} Rationale: \emph{Derives from Plasma requirements of section 3.5.4.1 of SSP 57003 Rev L.}

\end{itemize}

\textbf{[Excel Row 50]}: Acceleration - On-Orbit

\begin{quote}
The PCU shall meet structural integrity requirements in an On-Orbit acceleration environment having peak transient accelerations of up to 0.2 g, a vector quantity acting in any direction. These criteria are to be used as component load factors and assumes that the payload mass and center of gravity are within the envelope defined in Section K.3.1.7.3, Mass and Center of Gravity, (for Truss-based Payloads) or other appropriate payload carrier section (such as E.3.1.2, Mass and CG Capabilities, for FRAM-based Payloads or J.3.1, JCAP User Mass Properties, for JCAP Payloads). This acceleration is not intended to be used to calculate interface loads.
\end{quote}

\textbf{Requirement not flowed down: Assuming that PCU doesn't change configuration between installation and operations and operation has no effect on tolerance to acceleration, line \#52 encompasses this requirement.}

\textbf{[Excel Row 51]}: Acceleration - Installation

\begin{quote}
During installation, the PCU shall meet structural integrity requirements having peak transient accelerations of up to 0.4 g, a vector quantity in any direction. This criterion is to be used to assess the effects of the impact during payload installation. This acceleration is not intended to be used to calculate interface loads.
\end{quote}

[PCU.16400] Acceleration Environment

The PCU shall meet structural integrity requirements in an On-Orbit acceleration environment having peak transient accelerations of up to 0.4 g, a vector quantity acting in any direction.

\begin{itemize}
\item{} Rationale: \emph{This criterion is to be used to assess the effects of the impact during payload installation. This acceleration is not intended to be used to calculate interface loads. One ``g'' is equivalent to 9.8 m\slash s\textsuperscript{2}. Derives from Acceleration Environment requirements of section 3.5.9 of SSP 57003 Rev L.}

\end{itemize}

\textbf{[Excel Row 52]}: Data\slash Commanding

\begin{quote}
I don't know how that is being handled, I know they are sending data to the HISIE (tempertures)
\end{quote}

\textbf{Requirement not flowed down: Already captured in LaRC-generated requirements}

\textbf{[Excel Row 53]}: Discrete Output Logic Levels (from ExPCA)

\begin{quote}
The PCU shall recognize a digital “zero” level input (VIL) and digital “one” level input (VIH) for +5 V or +28 V logic levels from the ExPCA as defined in Table L.3.7.4.1.1--1, Discrete Output Logic Levels, between the payload input pins.
57003 Table L.3.7.4.1.1--1 Discrete Output Logic Levels
\end{quote}

[PCU.11000] Input Discrete Logic Level

The PCU shall recognize a digital logic zero (VIL) and digital logic one (VIH) for +5 V or +28 V \TBR{Voltage of Discrete Input Signals} logic levels from the ExPCA as defined in Table L.3.7.4.1.1--1, Discrete Output Logic Levels, between the payload input pins.

\begin{itemize}
\item{} Rationale: \emph{The discrete signal conveys the power on command to the PCU. Document No. 154920 identifies the pins, part number, manufacturer, and description of the Discrete Enable Input interface jack and mating connector. Derives from Acceleration Environment requirements of section L.3.7.4.1.1 of SSP 57003 Rev L.}

\end{itemize}

\textbf{[Excel Row 54]}: Discrete Input Maximum and Minimum Fault Voltages

\begin{quote}
The PCU shall not output any voltage < -0.2 Vdc or ≥ +30.6 Vdc due to any fault condition.
\end{quote}

\textbf{Requirement not flowed down: N\slash A because PCU not utilizing output discretes back to ISS}

\textbf{[Excel Row 55]}: Payload Discrete Loading Requirements

\begin{quote}
A. The ELC Payload shall not exceed the maximum capacitive (1 μF) and inductive (10 mH) loading on discrete outputs from the ExPCA.
B. The ELC Payload maximum resistive loading for the ExPCA +28 V outputs shall be no less than 700 ohms except when used for switch closure detection. The ExPCA discrete output voltage span of +28 Vdc +2.6\slash -4.0 Vdc is not guaranteed when loading exceeds 35mA (current-limiting will occur with FRAM-based Payload loads drawing \textsubscript{35} mA).
C. The ELC Payload maximum resistive loading for the ExPCA +5 V outputs shall be no less than 125 ohms, except when used for switch closure detection. The ExPCA discrete output voltage span of +5 Vdc +1\slash -2.6 Vdc is not guaranteed when loading exceeds 35 mA (current-limiting will occur with FRAM-based Payload loads drawing \textsubscript{35} mA or greater).
\end{quote}

[PCU.11032] Discrete Loading - Resistance 5 V

The PCU shall ensure that the maximum resistive loading on the ExPCA +5 discrete output exceeds 125 Ohm, except for when used for switch closure detection.

\begin{itemize}
\item{} Rationale: \emph{The ExPCA discrete output voltage span of +5 Vdc +1\slash -2.6 Vdc is not guaranteed when loading exceeds 35 mA (current-limiting will occur with FRAM-based Payload loads drawing \textsubscript{35} mA or greater). This derives from the Payload Discrete Loading Requirements in section L.3.7.4.6.C of SSP 57003 Rev L.}

\item{} Comments from CORE database: \textbf{This will be tailored depending on whether PCU uses 28 V or 5 V for discretes.}
[PCU.11025] Discrete Loading - Inductance

\end{itemize}

The PCU shall prevent the maximum inductive loading the ExPA discrete outputs from exceeding 10mH.

\begin{itemize}
\item{} Rationale: \emph{This derives from the Payload Discrete Loading Requirements in section L.3.7.4.6.A of SSP 57003 Rev L.}\\
[PCU.11020] Discrete Loading - Capacitance

\end{itemize}

The PCU shall prevent the maximum capacitive loading on the ExPA discrete outputs from exceeding 1 microFarad.

\begin{itemize}
\item{} Rationale: \emph{This derives from the Payload Discrete Loading Requirements in section L.3.7.4.6.A of SSP 57003 Rev L.}\\
[PCU.11031] Discrete Loading - Resistance 28 V

\end{itemize}

The PCU shall ensure that the maximum resistive loading on the ExPCA +28V discrete output exceeds 700 Ohm, except for when used for switch closure detection.

\begin{itemize}
\item{} Rationale: \emph{The ExPCA discrete output voltage span of +28 Vdc +2.6\slash -4.0 Vdc is not guaranteed when loading exceeds 35mA (current-limiting will occur with FRAM-based Payload loads drawing \textsubscript{35} mA). This derives from the Payload Discrete Loading Requirements in section L.3.7.4.6.B of SSP 57003 Rev L.}

\item{} Comments from CORE database: \textbf{This will be tailored depending on whether PCU used 28 V or 5 V for discretes.}

\end{itemize}

\textbf{[Excel Row 56]}: Environmental Compatibility

\begin{quote}
A payload shall be certified safe in the applicable worst case natural and induced environments defined in the Payload Integration Agreement (PIA) and\slash or Interface Control Document (ICD).
\end{quote}

\textbf{Requirement not flowed down: Applies to CPRSP, not to PCU.}

\textbf{[Excel Row 57]}: General

\begin{quote}
Electrical power distribution circuitry shall be designed to include circuit protection devices to protect against damage normally associated with an electrical fault when such a fault could result in damage to the visiting launch vehicle\slash ISS or present a hazard to the crew by direct of propagated effects. Circuit protective devices and wire sizes shall conform to the requirements of SPP 57000, Pressurized Payload Interface Requirements Document (IRD) or ISS approved International Partner equivalent, at the payload power distribution interface with ISS. For payload power distribution within the payload, circuit protective devices shall be sized such that steady state currents in excess of those allowed by TM 102179, Selection of Wires and Circuit Protective Devices of Orbiter Vehicle Payload Electrical Circuits are precluded. Bent pins or conductive contamination in an electrical connector will not be considered a credible failure mode if a post mate functional verification is performed to assure that shorts between adjacent connector pins or from pins to connector shell do not exist. If this test cannot be performed, then the electrical design must insure that any pin if bent prior to or during connector mating cannot invalidate more than one inhibit and that conductive contamination is precluded by proper inspection procedures.
\end{quote}

[PCU.19500] Safety - Circuit Protection 1

PCU shall incorporate circuit protection devices to protect against damage normally associated with an electrical fault when such a fault could result in damage to the visiting launch vehicle\slash ISS or present a hazard to the crew by direct of propagated effects

\begin{itemize}
\item{} Rationale: \emph{Derives from Electrical Systems General section 3.14.1 of SSP 51700.}

\item{} Comments from CORE database: \textbf{\textbf{Not in AVM. Is this a unique applicable requirement?}}
[PCU.19505] Safety - Circuit Protection 2

\end{itemize}

PCU shall incorporate circuit protective devices that comply with SSP 57000, Pressurized Payload Interface Requirements Document (IRD) or ISS approved International Partner equivalent, at the payload power distribution interface with ISS.

\begin{itemize}
\item{} Rationale: \emph{Derives from Electrical Systems General section 3.14.1 of SSP 51700.}

\item{} Comments from CORE database: \textbf{\textbf{Not in AVM. Is this a unique applicable requirement?}}
[PCU.19510] Safety - Wire Sizing

\end{itemize}

The PCU shall incorporate wire sizes that conform to the requirements of SSP 57000, Pressurized Payload Interface Requirements Document (IRD) or ISS approved International Partner equivalent, at the payload power distribution interface with ISS.

\begin{itemize}
\item{} Rationale: \emph{Derives from Electrical Systems General section 3.14.1 of SSP 51700.}

\item{} Comments from CORE database: \textbf{\textbf{Not in AVM. Is this a unique applicable requirement?}}
[PCU.19509] Safety - Circuit Protection 3

\end{itemize}

The PCU shall incorporate circuit protective devices sized such to preclude steady state currents in excess of those allowed by TM 102179, Selection of Wires and Circuit Protective Devices of Orbiter Vehicle Payload Electrical Circuits are precluded.

\begin{itemize}
\item{} Rationale: \emph{Derives from Electrical Systems General section 3.14.1 of SSP 51700.}

\item{} Comments from CORE database: \textbf{\textbf{Not in AVM. Is this a unique applicable requirement?}}

\end{itemize}

\textbf{[Excel Row 58]}: Lightning

\begin{quote}
CPRSP electrical circuits may be subject to the electromagnetic fields described in NSTS 21000-IDD-ISS due to a lightning strike to the launch pad. If circuit upset could result in a catastrophic hazard to the visiting launch vehicle, the circuit design shall be hardened against the environment or insensitive devices (relays) shall be added to control the hazard.
\end{quote}

\textbf{Requirement not flowed down: Don’t recall having sent this to LASP. Recommend drop.}

\textbf{[Excel Row 59]}: Electrical Connections

\begin{quote}
The design of electrical connectors shall make it impossible to inadvertently reverse a connection or mate the wrong connectors if a hazardous condition can be created. Payload and on-orbit support equipment wire harnesses and connectors shall be designed such that no blind connectors or disconnections must be made during payload installation, operation, removal, or maintenance on-orbit unless the design includes scoop proof connectors or other protective features.
\end{quote}

\textbf{Requirement not flowed down: Overcome by specification of connectors that LASP sent.}

\textbf{[Excel Row 60]}: Launch Loads

\begin{quote}
The PCU shall survive axial random vibrationn loads of TBR
\end{quote}

[PCU.16410] Launch Loads - Axial

The PCU shall survive axial random vibration loads of \TBD{Axial Launch Load Limit}.

\begin{itemize}
\item{} Rationale: \emph{Requirement levied by LASP via spreadsheet dated 05MAR2018.}

\end{itemize}

\textbf{[Excel Row 61]}: Launch Loads

\begin{quote}
The PCU shall survive lateral random vibrationn loads of TBR
\end{quote}

[PCU.16415] Launch Loads - Lateral

The PCU shall survive lateral random vibration loads of \TBD{Lateral Launch Load Limit}.

\begin{itemize}
\item{} Rationale: \emph{Requirement levied by LASP via spreadsheet dated 05MAR2018.}

\end{itemize}

\textbf{[Excel Row 62]}: Testing

\begin{quote}
The PCU shall be tested to random vibration loads for a duration of 1 minute per axis
\end{quote}

[PCU.20005] Testing - Random Vibration

The PCU shall be tested to random vibration loads for a duration of 1 minute per axis

\begin{itemize}
\item{} Rationale: \emph{Requirement levied by LASP via spreadsheet dated 05MAR2018.}

\item{} Comments from CORE database: \textbf{\textbf{Is this a unique requirement, or a verification requirement linked to some of the acceleration requirements in PCU.164XX?}}

\end{itemize}

\textbf{[Excel Row 63]}: Minimum Natural Frequency

\begin{quote}
The PCU shall have a minimum natural frequency of 250hz TBR
\end{quote}

[PCU.16425] Minimum Natural Frequency

The PCU shall have a minimum natural frequency of 250 Hz \TBD{Minimum Natural Frequency}.

\begin{itemize}
\item{} Rationale: \emph{Requirement levied by LASP via spreadsheet dated 05MAR2018.}

\end{itemize}

\textbf{[Excel Row 64]}: FRAM Payload Orientations

\begin{quote}
Payloads in the ground processing configuration [with Remove Before Flight (RBF) items attached] shall be retained and maintain positive margins of safety and meet specified performance requirements when exposed to the following physical operations and resulting orientations, including all intermediary orientations and any prolonged exposure to orientations: • Full 180 degree rotation from right-side up to upside down (flight orientation).
• 90 degree orientation from right-side up during horizontal ground operations with the launch vehicle.
\end{quote}

\textbf{Requirement not flowed down: Payload level requirement not applicable to PCU}

\textbf{[Excel Row 65]}: Acoustics

\begin{quote}
PCU enclosure components shall have a ``break even'' value of 150in\textsuperscript{2}\slash lb or less.
\end{quote}

[PCU.16430] Acoustics

PCU enclosure components shall have a ``break even'' value of 150in\textsuperscript{2}\slash lb or less.

\begin{itemize}
\item{} Rationale: \emph{Requirement levied by LASP via spreadsheet dated 05MAR2018}

\item{} Comments from CORE database: \textbf{\textbf{Follow up with LASP to understand what this means.}}

\end{itemize}

\textbf{[Excel Row 66]}: Thermal Testing

\begin{quote}
The PCU shall be tested in a thermal vacuum chamber to XXX temperatures TBR
\end{quote}

[PCU.20010] Testing - Thermal

The PCU shall be tested in a thermal vacuum chamber to \TBD{Thermal Testing Temperature}.

\begin{itemize}
\item{} Rationale: \emph{Requirement levied by LASP via spreadsheet dated 05MAR2018.}

\item{} Comments from CORE database: \textbf{Is this a unique requirement, or a verification requirement linked to PCU.162XX, PCU.181XX, or additional requirements?}

\end{itemize}

\textbf{[Excel Row 67]}: Structural Loading

\begin{quote}
The PCU shall be able to survive a static load of 60g applied in either X, Y or Z.
\end{quote}

[PCU.16420] Structural Loading

The PCU shall survive a static load of 60g applied in each of the X, Y, and Z axes in the ExPA Reference Coordinate System.

\begin{itemize}
\item{} Rationale: \emph{Requirement levied by LASP via spreadsheet dated 05MAR2018. Reference: D683--97497--01. One ``g'' is equivalent to 9.8 m\slash s\textsuperscript{2}.}

\end{itemize}

\textbf{[Excel Row 68]}: Factors of Safety

\begin{quote}
nan
\end{quote}

\textbf{Requirement not flowed down: Test Requirement? Not sure how to implement.}

\input{mmd6-article-footer}

\end{document}
